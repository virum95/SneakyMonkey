
%----------------------------------------------------------------------------------------
%	PACKAGES AND OTHER DOCUMENT CONFIGURATIONS
%----------------------------------------------------------------------------------------

\documentclass[paper=a4, fontsize=12pt]{scrartcl} % A4 paper and 11pt font size

%\usepackage{fourier} % Use the Adobe Utopia font for the document - comment this line to return to the LaTeX default
\usepackage[spanish]{babel} % English language/hyphenation

\usepackage{amsmath,amsfonts,amsthm} % Math packages
\usepackage[utf8]{inputenc}

\usepackage{lipsum} % Used for inserting dummy 'Lorem ipsum' text into the template

\usepackage{sectsty} % Allows customizing section commands
\allsectionsfont{\sffamily\scshape} % Make all sections centered, the default font and small caps

\usepackage{fancyhdr} % Custom headers and footers
\pagestyle{fancyplain} % Makes all pages in the document conform to the custom headers and footers
\fancyhead{} % No page header - if you want one, create it in the same way as the footers below
\fancyfoot[L]{} % Empty left footer
\fancyfoot[C]{} % Empty center footer
\fancyfoot[R]{\thepage} % Page numbering for right footer
\renewcommand{\headrulewidth}{0pt} % Remove header underlines
\renewcommand{\footrulewidth}{0pt} % Remove footer underlines
\setlength{\headheight}{13.6pt} % Customize the height of the header

\numberwithin{equation}{section} % Number equations within sections (i.e. 1.1, 1.2, 2.1, 2.2 instead of 1, 2, 3, 4)
\numberwithin{figure}{section} % Number figures within sections (i.e. 1.1, 1.2, 2.1, 2.2 instead of 1, 2, 3, 4)
\numberwithin{table}{section} % Number tables within sections (i.e. 1.1, 1.2, 2.1, 2.2 instead of 1, 2, 3, 4)

\setlength\parindent{0pt} % Removes all indentation from paragraphs - comment this line for an assignment with lots of text

%----------------------------------------------------------------------------------------
%	TITLE SECTION
%----------------------------------------------------------------------------------------

\newcommand{\horrule}[1]{\rule{\linewidth}{#1}} % Create horizontal rule command with 1 argument of height

\title{	
\normalfont \normalsize 
\textsc{Universidad de Deusto} \\ [25pt] % Your university, school and/or department name(s)
\horrule{0.5pt} \\[0.4cm] % Thin top horizontal rule
\huge Justificación y descripción de plataforma de microdonaciones para alboan\\ % The assignment title
\horrule{2pt} \\[0.5cm] % Thick bottom horizontal rule
\author{Gaizka Virumbrales \& Rubén Sánchez} % Your name
}



\date{\normalsize\today} % Today's date or a custom date

\begin{document}

\maketitle % Print the title

%----------------------------------------------------------------------------------------
%	PROBLEM 1
%----------------------------------------------------------------------------------------

\section{Descripción del proyecto}

El proyecto consiste en la creación de un widget para el apoyo a los proyectos de Alboan mediante microdonaciones.\\
Este widget permitirá a las diferentes paginas web añadir en sus carros de la compra la opción de donar una pequeña cantidad de dinero a una de los diferentes proyectos que lleva Alboan.\\
Todo el proyecto estará soportado por una base de datos en la que guardaremos la información de las personas que han hecho una aportación. Con la información de estas podremos continuar el contacto con ellas y crear perfiles de donantes.

%------------------------------------------------

\subsection{El widget}
Las empresas que decidan participar con este widget, tendrán que añadir unas simples líneas de código en su página web que inyectarán automáticamente el contenido. Este widget es \textit{responsive}, se carga de forma asíncrona y es personalizable por la empresa en cuestión. Además, contará con un soporte para múltiples idiomas.\\
El código tendrá que estar en dos páginas web: en la anterior al pago o \textit{checkout}, en la que la empresa hace una petición para poner el widget en su sitio web y el la de la confirmación de la compra, en la que la empresa se hace responsable de enviar la información relativa a la donación, en caso de que se haya producido.

\section{Justificación}
La justificación de este proyecto es solo proveer al donante una plataforma sencilla, rápida y cómoda de colaborar con proyectos solidarios. Además, también pretende alcanzar a personas que de otra forma no estarían enteradas de este proyecto. \\
Dependiendo del tipo de empresa que contrate el servicio, se pueden ofrecer uno u otro proyecto de Alboan. Por ejemplo, si la empresa es del sector de la tecnología se podría ofrecer el proyecto de \textit{Tecnología libre de conflicto}. De esta forma, se logra informar a las personas de los proyectos en los que más interesados podrían estar.

\section{Alcance del proyecto}


\end{document}